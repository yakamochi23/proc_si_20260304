\documentclass[../main.tex]{subfiles} % 親ファイルを参照


\setcounter{chapter}{0} % 章番号を付番
\begin{document}

    \chapter{序論}
        ここに章の内容を書きます.序論を書く.\cite{Simonyan15}

        画像を載せたい場合は,
        \begin{figure}[H]
            \centering
            \includegraphics[width=70mm]{figures/chapter1/mucha.png}
            \caption{アルフォンス・ミュシャ}
            \label{fig:mucha}
        \end{figure}
        のようにして画像を挿入することができる.図\ref{fig:mucha}は肖像画である.

        2つの画像を横並びで挿入する.図\ref{fig:彩度を半分にする前後の画像}を見ると,源画像(図\ref{fig:彩度を半分にする前の画像})の彩度が低減された画像(図\ref{fig:彩度を半分にした後の画像})が得られていることが確認できる.
        \begin{figure}[H]
            \centering
            % 左側の画像
            \begin{subfigure}[b]{.5\linewidth}
                \centering
                \includegraphics[width=0.8\linewidth]{figures/sample_fig/lena.png}
                \subcaption{源画像}
                \label{fig:彩度を半分にする前の画像}
            \end{subfigure}
            % 右側の画像
            \begin{subfigure}[b]{.5\linewidth}
                \centering
                \includegraphics[width=0.8\linewidth]{figures/sample_fig/lena_half.png}
                \subcaption{彩度を半分にした画像}
                \label{fig:彩度を半分にした後の画像}
            \end{subfigure}
            \caption{彩度を半分にする前後の画像}
            \label{fig:彩度を半分にする前後の画像}
        \end{figure}

        数式を書いてみる.
        \begin{align}
            y = f(x)
            \label{eq:sample}
        \end{align}
        式\ref{eq:sample}は$y$が$x$の関数であることを表す.

    % 文献情報を条件分岐(サブファイルが直接コンパイルされた場合のみ適用)
    \ifSubfilesClassLoaded{
        \bibliography{bibliography/reference} % BibTeX ファイル名を指定(拡張子は不要)
        \bibliographystyle{junsrt} % 文献スタイルを指定
    }{}

\end{document}