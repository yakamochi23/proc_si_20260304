%%
%% 研究報告用スイッチ
%% [techrep]
%%
%% 欧文表記無しのスイッチ(etitle,eabstractは任意)
%% [noauthor]
%%




%\usepackage[dvips]{graphicx}
\usepackage[dvipdfmx]{graphicx}
\usepackage{latexsym}

\def\Underline{\setbox0\hbox\bgroup\let\\\endUnderline}
\def\endUnderline{\vphantom{y}\egroup\smash{\underline{\box0}}\\}
\def\|{\verb|}
%

%\setcounter{巻数}{59}%vol59=2018
%\setcounter{号数}{10}
%\setcounter{page}{1}

\AtBeginDvi{\special{pdf:mapfile uptex-haranoaji.map}}
\AtBeginDvi{\special{pdf:mapline rml  UniJIS2004-UTF16-H HaranoAjiMincho-Regular.otf}}
\AtBeginDvi{\special{pdf:mapline rmlv UniJIS2004-UTF16-V HaranoAjiMincho-Regular.otf}}
\AtBeginDvi{\special{pdf:mapline gbm  UniJIS2004-UTF16-H HaranoAjiGothic-Medium.otf}}
\AtBeginDvi{\special{pdf:mapline gbmv UniJIS2004-UTF16-V HaranoAjiGothic-Medium.otf}}

\usepackage{amsmath,amssymb,bm}% 数式用
\usepackage{subfiles}
\renewcommand{\baselinestretch}{1.0} % 全体の行間を変更(デフォルトは倍率1.0, 先生に提出する際は2.0くらいが良い)
\usepackage[subrefformat=parens]{subcaption}% 複数の図を貼るときのsubcaption
\renewcommand\thesubfigure{(\alph{subfigure})}% 図版のフォーマット用
\captionsetup[subfigure]{labelformat=simple, labelsep=space}% 図版のフォーマット用