\documentclass[../main.tex]{subfiles} % 親ファイルを参照


\begin{document}
        
    %4
    \section{評価実験}
    
        \subsection{実験条件}
            ここで実験条件を述べる.
                \begin{itemize}
                    \item データセット
                    \item 評価指標
                    \item YOLO-X
                    \item HR-Net,取り出す層
                    \item 各種パラメータの値
                \end{itemize}


        \subsection{実験結果および考察}

        \subsection{評価データセット}
            本研究では,評価データセットとして,アイスホッケー選手を追跡対象とするVIP-HTDデータセット\cite{}を用いた.

        \subsection{評価指標}
            本研究では,評価指標として,MOTA(Multi-Object Tracking Accuracy)を用いた.
                \begin{equation}
                    MOTA = 1 - \frac{|FN| + |FP| + |IDs|}{|GT|}
                    \label{eq:MOTA}
                \end{equation}
            ここで,$|FN|$は,

            また,$IDF1$は,$IDP$と$IDR$の調和平均として式\ref{eq:IDF1}で定義される.
                \begin{equation}
                    IDF1 = \frac{2 \times IDP \times IDR}{IDP + IDR}
                    \label{eq:IDF1}
                \end{equation}
            ここで,

    % 文献情報を条件分岐(サブファイルが直接コンパイルされた場合のみ適用)
    \ifSubfilesClassLoaded{
        \bibliography{bibliography/sample} % BibTeX ファイル名を指定(拡張子は不要)
        \bibliographystyle{ipsjunsrt} % 文献スタイルを指定
    }{}

\end{document}