\documentclass[../main.tex]{subfiles} % 親ファイルを参照


\begin{document}
        
    %4
    \section{評価実験}
    
        \subsection{実験条件}
            ここで実験条件を述べる.
                \begin{itemize}
                    \item データセット
                    \item 評価指標
                    \item YOLO-X
                    \item HR-Net,取り出す層
                    \item 各種パラメータの値
                \end{itemize}


        

        \subsection{評価データセット}
            評価実験では,試合中のアイスホッケー選手を追跡対象とするVIP-HTDデータセット\cite{}を評価に用いた.VIP-HTDは,

        \subsection{評価指標}
            評価実験では,$\mathrm{MOTA}$(Multi-Object Tracking Accuracy)\cite{MOTA},$\mathrm{IDF1}$(ID F1-score)\cite{IDF1}を評価指標として用いた.
            $\mathrm{MOTA}$は$|\mathrm{TP}|,|\mathrm{FP}|,|\mathrm{FN}|,|\mathrm{IDs}|$から算出されるMOTの総合評価指標であり,追跡精度を物体のBBox単位で評価する.ここで$|\mathrm{TP}|$は推定BBoxと真値の一致数,$|\mathrm{FP}|$は真値と一致しなかった推定BBoxの数,$|\mathrm{FP}|$は推定BBoxに一致しなかった真値の数,$|\mathrm{IDs}|$はID Switchの総発生回数を表す.
            また,$\mathrm{IDF1}$は$\mathrm{IDP}$と$\mathrm{IDR}$の調和平均として算出されるMOTの総合評価指標であり,追跡精度を物体の軌跡単位で評価する.ここで,$\mathrm{IDP}$と$\mathrm{IDR}$は推定軌跡と真値の間で算出される適合率・再現率を表す.

        \subsection{実験結果および考察}





    % 文献情報を条件分岐(サブファイルが直接コンパイルされた場合のみ適用)
    \ifSubfilesClassLoaded{
        \bibliography{bibliography/sample} % BibTeX ファイル名を指定(拡張子は不要)
        \bibliographystyle{ipsjunsrt} % 文献スタイルを指定
    }{}

\end{document}