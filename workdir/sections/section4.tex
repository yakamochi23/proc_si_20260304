\documentclass[../main.tex]{subfiles} % 親ファイルを参照


\begin{document}
        
    %4
    \section{評価実験}
    
        \subsection{実験条件}

            本研究では,VIP-HTD(Vision and Image Processing Hokey Tracking Dataset)\cite{}を用いて,従来手法であるOC-SORTと提案手法の比較評価を行った.この際,OC-SORTと提案手法の両方にByteTrack\cite{ByteTrack2022}で提案された検出信頼度に基づいた二段階のAssociationを導入した.これは,スポーツシーンにおいてはカメラモーションによるMotion Blurが頻発し,検出信頼度の低い検出結果が得られる傾向にあることを考慮してのことである.VIP-HTDは,試合中のアイスホッケー選手を追跡対象とするMOTデータセットであり,計22個(訓練:14個/評価:1個/テスト:7個)のシーケンス,計73,890枚(訓練:45,396枚/評価:2,981枚/テスト:25,513枚)のフレームで構成されている.

            物体検出器には,VIP-HTDの訓練データで事前学習済みのYOLOX\cite{YOLOX2021}を使用した.物体追跡では,YOLOXが出力した検出結果の内,検出信頼値$c_{d} \in [0,1]$が$0.6 \le c_{d} \le 1.0$のものを第一段階のAssociationに,$0.1 < c_{d} < 0.6$のものを第二段階のAssociationに利用した.また,姿勢推定器にはMS COCOで事前学習済みのHRNet(High-Resolution Network)\cite{HRNet2019}を用い,物体追跡ではHRNetのlayer4が出力する特徴マップから得られた姿勢特徴量$\mathbf{f} \in \mathbb{R}^{48}$をAssociationに利用した.

            評価実験では,$\mathrm{MOTA}$(Multi-Object Tracking Accuracy)\cite{MOTA},$\mathrm{IDF1}$(ID F1-score)\cite{IDF1}を評価指標として用いた.
            $\mathrm{MOTA}$は$|\mathrm{TP}|,|\mathrm{FP}|,|\mathrm{FN}|,|\mathrm{IDs}|$から算出されるMOTの総合評価指標であり,追跡精度を物体のBBox単位で評価する.ここで$|\mathrm{TP}|$は推定BBoxと真値の一致数,$|\mathrm{FP}|$は真値と一致しなかった推定BBoxの数,$|\mathrm{FP}|$は推定BBoxに一致しなかった真値の数,$|\mathrm{IDs}|$はID Switchの総発生回数を表す.
            また,$\mathrm{IDF1}$は$\mathrm{IDP}$と$\mathrm{IDR}$の調和平均として算出されるMOTの総合評価指標であり,追跡精度を物体の軌跡単位で評価する.ここで,$\mathrm{IDP}$と$\mathrm{IDR}$は推定軌跡と真値の間で算出される適合率・再現率を表す.

        \subsection{実験結果および考察}
            表\ref{}に,VIP-HTDデータセットのテストデータに対する定量評価結果を示す.

            図\ref{}に,定性評価結果を示す.





    % 文献情報を条件分岐(サブファイルが直接コンパイルされた場合のみ適用)
    \ifSubfilesClassLoaded{
        \bibliography{bibliography/sample} % BibTeX ファイル名を指定(拡張子は不要)
        \bibliographystyle{ipsjunsrt} % 文献スタイルを指定
    }{}

\end{document}