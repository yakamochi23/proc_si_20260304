\documentclass[../main.tex]{subfiles} % 親ファイルを参照


\begin{document}
        
    \begin{abstract}
        スポーツシーンでは選手が高速かつ不規則に動く上に,ユニフォームの共通性により外観の識別性が低下しやすい.
        そのため,位置や外観の類似性を手掛かりに物体を追跡する従来手法は,異なる物体同士を誤って対応付けてしまう可能性がある.
        本研究では,姿勢情報を考慮した複数人物追跡手法を提案し,スポーツ選手を対象としたMOTの高精度化に取り組んだ.外観が類似する状況でも人物の識別性を強化するため,姿勢推定器の中間特徴量の類似性をAssociationに導入した.
        評価実験では,アイスホッケー選手を追跡対象とするVIP-HTDデータセットを用い,MOTAやIDF1などの評価指標により提案手法の有効性を検証した.
    \end{abstract}
            
    % 文献情報を条件分岐(サブファイルが直接コンパイルされた場合のみ適用)
    \ifSubfilesClassLoaded{
        \bibliography{bibliography/sample} % BibTeX ファイル名を指定(拡張子は不要)
        \bibliographystyle{ipsjunsrt} % 文献スタイルを指定
    }{}

\end{document}