\documentclass[../main.tex]{subfiles} % 親ファイルを参照


\begin{document}

    \section{投稿の流れ}


        %2.1
        \subsection{準備}

            情報処理学会論文誌ジャーナルの\LaTeX スタイルファイルを含む論文執筆キットは
                \begin{quote}
                \small
                \|http://www.ipsj.or.jp/jip/submit/style.html|
                \end{quote}
            からダウンロードすることができる.論文執筆キットは以下のファイルを含んで
            いる.
                \begin{enumerate}
                \item \|ipsj.cls      |: 最終原稿用スタイルファイル
                \item \|ipsjdraft.sty |: 投稿用スタイル(査読用)
                \item \|ipsjpref.sty  |: 序文用スタイル
                \item \|jsample.tex   |: 本稿のソースファイル
                \item \|esample.tex   |: 英文サンプルのソースファイル
                \item \|ipsjsort.bst  |: jBibTEX スタイル(著者名順)
                \item \|ipsjunsrt.bst |: jBibTEX スタイル(出現順)
                \item \|bibsample.bib |: 文献リストのサンプル
                \item \|ebibsample.bib|: 英文文献リストのサンプル
                \item \|tech-jsample.tex:| 研究報告(和文)のサンプル
                \item \|tech-esample.tex:| 研究報告(英文)のサンプル
                \end{enumerate}
            実行環境としては\LaTeXe を前提としているので,準備されたい.







        %2.2
        \subsection{原稿の作成と投稿}

            本稿に従って用意した投稿用原稿の\LaTeX ソースからpdfファイルを作成し,
            Adobeのpdf readerで読めることを確認した後,
                \begin{quote}
                \small
                \|https://ipsj1.i-product.biz/ipsjsig/**|
                \end{quote}
            (**部分は研究会の略称,DBS等)の研究会投稿システムにて,指示にし従い投稿する.


    % 文献情報を条件分岐(サブファイルが直接コンパイルされた場合のみ適用)
    \ifSubfilesClassLoaded{
        %\bibliography{bibliography/bibsample,bibliography/ebibsample} % BibTeX ファイル名を指定(拡張子は不要)
        %\bibliography{bibliography/reference}
        \bibliography{bibliography/sample}
        \bibliographystyle{junsrt} % 文献スタイルを指定
    }{}

\end{document}