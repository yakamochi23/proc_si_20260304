\documentclass[../main.tex]{subfiles} % 親ファイルを参照


\begin{document}

    \section{関連研究}

        MOTを解く代表的なパラダイムは``Tracking-by-Detection''である.このパラダイムでは,まず各フレームに登場する物体を検出し,その後,既存のトラックと検出結果を何らかの指標に基づいて対応付ける.これにより,異なるフレーム間で同じ物体には同じ識別子(ID)が割り当てられ,時間的な物体追跡が実現する.ここで,既存トラックと検出結果の間で同一物体を対応付ける操作を関連付け(Association)と呼ぶ.Associationは,トラックの集合を $\hat{\mathcal{X}}=\{\hat{\mathbf{x}}_i\}_{i=1}^{M}$,検出結果の集合を $\mathcal{Y}=\{\mathbf{y}_j\}_{j=1}^{N}$ とすると,両集合の要素を重複なく1対1で対応付け,コスト関数$\mathcal{L}_\mathrm{match}(\cdot)$の総和を最小化するインデックス割当 $\hat{P}$ を求める問題に帰着する.これは,式(\ref{eq:Association})のように線形割当問題として定式化され,ハンガリアン法\cite{Hungarian1955}などのアルゴリズムによって解かれる.
            \begin{equation}
                \hat{P} = \underset{P \in \mathcal{P}} {\operatorname{argmin}} \sum_{i=1}^{M}\sum_{j=1}^{N}\mathcal{L}_\mathrm{match}(\hat{\mathbf{x}}_{i}, \mathbf{y}_{j})
                \label{eq:Association}
            \end{equation}

        SORT\cite{SORT2016}はTracking-by-Detectionに基づく代表的な追跡手法である.SORTでは,物体の位置とその速度によって状態変数を記述し,Kalman Filter\cite{kalmanfilter1960}を用いて状態の予測・更新を行う.この際,Associationのコストには,IoU(Intersection over Union)が用いられる.IoUは2つの図形の重なり度合いを評価する指標であり,SORTでは前時刻から状態予測されたトラックと現在の検出結果のBounding-Box(BBox)間でIoUを計算し,物体の位置的類似性に基づいたAssociationを行う.SORTは,非常にシンプルな枠組みで物体追跡を実現する一方で,複雑に動く物体の追跡を苦手とする.これは,Kalman Filterが前提とする状態空間が,状態の遷移に線形性を仮定することに起因する.物体が不規則な動きを示したり,カメラモーションが加わると,物体の見かけの動きが複雑化し,線形運動を仮定した状態予測は,予測トラックと検出結果の位置的類似性を低下させる.その結果,IoUのみに基づいたAssociationではID Switch(フレーム間で同一物体のIDが切り替わる現象)などの誤追跡が生じやすくなる.これに対して,物体の非線形運動や社会性を考慮した高精度な予測器\cite{pargiclefilter1993,SocialLSTM2016,SocialTransmotion2024}が提案されているが,いずれも追跡に要する計算コストが増大することから,現在でもKalman Filterによる状態予測がMOTでは主流となっている.

        DeepSORT\cite{DeepSORT2017}では,SORTの課題を克服することを目的に,IoUに加えて物体の外観情報をAssociationに利用している.具体的には,各検出物体のBBoxに対応する画像領域から,深層学習に基づく再同定器によって物体の外観特徴量を抽出し,その特徴量の類似性に基づいて既存トラックと検出結果を対応付ける.これにより,複雑な動きを示す物体に対しても頑健な物体追跡を実現している.しかし,物体の外観類似性に基づくAssociationは必ずしもMOTに有効であるとは限らない.例えば,外観識別性の高い物体が登場するシーン\cite{MOT16}においてはDeepSORTが高い追跡精度を示す一方で,選手が共通のユニフォームを着用するスポーツシーン\cite{DanceTrack2022}では,その追跡精度がSORTよりも劣ることが報告されている.

        近年,外観情報に依存せず,追跡精度の向上を図る手法も提案されている.例えば,OC-SORT\cite{OC-SORT2023}では,Occlusion等により一時的にダミー更新されたトラックが再び検出結果と対応付けられた際,過去の検出結果を参照してトラックの状態変数を再更新することで,Kalman Filterの状態推定誤差を緩和させている.
        また,姿勢情報をAssociationに利用した手法も提案されている\cite{PoseFlow2018,LightTrack2020}.
        これらの手法は,物体の身体構造を姿勢推定器が出力する推定関節点(keypoint)によって捉えることで,外観が類似する物体に対しても識別性を強化している.しかし,スポーツシーンでは,選手同士の接触や交錯によるOcclusionが頻発し,keypointの欠落・誤推定が生じやすい.その結果,keypointの類似度は同一人物間で不安定となり,Associationの精度が低下する恐れがある.さらに,keypointベースの追跡は高精度な姿勢推定器に大きく依存し,対象ドメインに特化した姿勢推定器の追加学習が必要となる.
        これにより,keypointアノテーション付きデータの収集コストも課題となる.
        
        本研究では,姿勢情報の有用性を活かしつつ,keypointの完全性に強く依存しない表現として,姿勢推定器の中間特徴量の類似性に基づくAssociationを提案する.これにより,外観識別性の低い物体が複雑に動き,Occlusionが多発するスポーツシーンにおいても安定した追跡を目指す.

    % 文献情報を条件分岐(サブファイルが直接コンパイルされた場合のみ適用)
    \ifSubfilesClassLoaded{
        \bibliography{bibliography/bib_si} % BibTeX ファイル名を指定(拡張子は不要)
        \bibliographystyle{ipsjunsrt} % 文献スタイルを指定
    }{}

\end{document}