\documentclass[../main.tex]{subfiles} % 親ファイルを参照


\begin{document}

    \section{関連研究}

        MOTを解く代表的なパラダイムはTracking-by-Detectionである.このパラダイムでは,まず時間的に連続する各フレームから物体を検出し,その後検出された同一物体に対して同じ識別子(ID)を割り当てることで追跡を行う.この同一物体を時間的に結びつける操作をAssociationと呼ぶ.図\ref{}にAssociationの例を示す.

        ある時刻において,トラックの集合を $\mathcal{X}=\{\mathbf{x}_i\}_{i=1}^{M}$,検出結果の集合を $\hat{\mathcal{Y}}=\{\hat{\mathbf{y}}_j\}_{j=1}^{N}$ とすると,Association は両集合の要素を重複なく1対1で対応付け,コスト$\mathcal{L}_\mathrm{match}(\cdot)$の総和を最小化する割当 $\hat{P}$ を求める問題に帰着する.これは式\ref{eq:Association}で表される線形割当問題として定式化され,ハンガリアン法\cite{<ref>} などにより解かれる.
            \begin{equation}
                \hat{P} = \underset{P \in \mathcal{P}_{M\times N}} {\operatorname{argmin}} \sum_{i=1}^{M}\sum_{j=1}^{N}\mathcal{L}_\mathrm{match}(\mathbf{x}_{i}, \hat{\mathbf{y}}_{j})
                \label{eq:Association}
            \end{equation}

        SORT\cite{SORT2016}は,Tracking-by-Detectionに基づく代表的手法である.SORTでは,各トラックの状態としてバウンディングボックス(BBox)とその速度を扱い,カルマンフィルタ\cite{}により状態の予測・更新を行う.この際,時刻$t$の更新ステップでは,時刻$t-1$
        IoU(Intersection over Union)を

        DeepSORT\cite{}は,SORTのAssociationに

        OC-SORTでは

        PoseTrack

    % 文献情報を条件分岐(サブファイルが直接コンパイルされた場合のみ適用)
    \ifSubfilesClassLoaded{
        \bibliography{bibliography/sample} % BibTeX ファイル名を指定(拡張子は不要)
        \bibliographystyle{ipsjunsrt} % 文献スタイルを指定
    }{}

\end{document}