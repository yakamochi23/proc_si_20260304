\documentclass[../main.tex]{subfiles} % 親ファイルを参照

\begin{document}

    \section{はじめに}

        複数物体追跡(Multi-Object Tracking, MOT)は,コンピュータビジョン分野における重要なタスクの1つであり,ビデオシーケンスに登場する複数の物体を一意に識別し,その動きを正確に追跡することを目的とする.
        
        近年,スポーツ領域においても運動解析や戦術分析への応用を背景として,試合中の選手を対象としたMOTの研究が注目されている.
        しかし,スポーツシーンは歩行者や車両を追跡対象とする一般的なMOTベンチマークと比べ,物体の追跡難易度が高いことが知られている.スポーツシーンでは,選手が高速かつ不規則に動くことに加え,チーム競技では共通のユニフォームを着用することで,外観の識別性が低下しやすい.その結果,物体の位置や外観の類似性に依存する従来の追跡手法は十分な性能を発揮できない場合がある.

        本研究では,スポーツ選手を対象としたMOTの高精度化を目的とし,姿勢情報を考慮した複数人物追跡手法を提案する.具体的には,試合中に選手が多様な姿勢をとる点に着目し,深層学習に基づく姿勢推定器の中間層から得られる特徴量に基づいて姿勢の類似性を定式化し,Associationに組み込むことで選手の識別性を強化する.これにより,複雑な運動や外観が類似する物体が登場するシナリオにおいても,安定した追跡が期待される.

            
    % 文献情報を条件分岐(サブファイルが直接コンパイルされた場合のみ適用)
    \ifSubfilesClassLoaded{
        \bibliography{bibliography/sample} % BibTeX ファイル名を指定(拡張子は不要)
        \bibliographystyle{ipsjunsrt} % 文献スタイルを指定
    }{}

\end{document}
