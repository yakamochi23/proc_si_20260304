\documentclass[../main.tex]{subfiles} % 親ファイルを参照

\begin{document}

    \section{はじめに}

        複数物体追跡(Multi-Object Tracking, MOT)は,コンピュータビジョン分野における重要なタスクの1つであり,ビデオシーケンスに登場する複数の物体を一意に識別し,その動きを正確に追跡することを目的とする.近年,スポーツ領域においても運動解析や戦術分析への応用を背景とし,試合映像中の選手を対象としたMOTの研究が注目されている.


        本研究では,スポーツ選手を対象としたMOTの高精度化を目的とし,姿勢情報を考慮した追跡手法を提案する.具体的には,
        これにより,
            
    % 文献情報を条件分岐(サブファイルが直接コンパイルされた場合のみ適用)
    \ifSubfilesClassLoaded{
        \bibliography{bibliography/sample} % BibTeX ファイル名を指定(拡張子は不要)
        \bibliographystyle{ipsjunsrt} % 文献スタイルを指定
    }{}

\end{document}
