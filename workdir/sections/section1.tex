\documentclass[../main.tex]{subfiles} % 親ファイルを参照

\begin{document}

    \section{はじめに}

        複数物体追跡(Multi-Object Tracking, MOT)は,コンピュータビジョン分野における重要なタスクの1つであり,ビデオシーケンスに登場する複数の物体を一意に識別し,その動きを正確に追跡することを目的とする.
        
        近年,スポーツ領域においても運動解析や戦術分析への応用を背景として,試合中の選手を対象としたMOTの研究が注目されている.
        しかし,スポーツシーンは歩行者や車両を追跡対象とする一般的なMOTベンチマークと比べて追跡難易度が高いことが知られている.スポーツシーンにおいては,選手が高速で不規則な動きを示すことが多い.また,チーム競技では各選手が共通したユニフォームを着用することで,その外観が類似しやすくなる.その結果,物体の位置や外観を手掛かりとして追跡を行う従来手法は,その性能を十分に発揮することができない.


        本研究では,スポーツ選手を対象としたMOTの高精度化を目的とし,姿勢情報を考慮した追跡手法を提案する.具体的には,試合中に選手が多様な姿勢をとる点に着目し,深層学習に基づく姿勢推定器の中間層から得られる特徴量に基づいて姿勢の類似性を定式化することで,物体の識別性を強化する.
        これにより,複雑な運動や類似した外観を持つ物体が登場するシナリオにおいても,安定したAssociationが期待される.
            
    % 文献情報を条件分岐(サブファイルが直接コンパイルされた場合のみ適用)
    \ifSubfilesClassLoaded{
        \bibliography{bibliography/sample} % BibTeX ファイル名を指定(拡張子は不要)
        \bibliographystyle{ipsjunsrt} % 文献スタイルを指定
    }{}

\end{document}
