\documentclass[../main.tex]{subfiles} % 親ファイルを参照


\begin{document}
        
    \section{提案手法}

        
        スポーツシーンでは,選手が急な加減速や方向転換を繰り返すため運動の不規則性が高く,さらにユニフォームの共通性により外観の識別性も低い.このため,IoUや外観特徴に依存した従来のAssociationには限界がある.一方,各選手は試合中に多様な姿勢をとり,隣接フレーム間では姿勢の連続性が保たれやすいことから,姿勢情報は有効な識別手掛かりとなり得る.しかし,keypointに基づくコスト関数は,Occlusion等に起因する欠落・誤推定の影響を受けやすく,Associationが不安定になる.
        そこで本研究では,keypointの完全性に過度に依存しない表現として,姿勢推定器の中間層から得られる特徴量を用い,姿勢の類似性を定式化してAssociationに組み込む手法を提案する.提案手法のパイプラインを図\ref{fig:method_overview}に示す.
            \begin{figure*}[tb]
                \centering
                \includegraphics[width=\linewidth]{figures/section3/method_overview.png}
                \caption{提案する姿勢情報に基づいたAssociationの概要.検出結果$\{\mathbf{y}_{j}\}_{j=1}^{M}$に対して姿勢推定器を用いた特徴抽出を独立に行い,姿勢特徴量$\{\mathbf{f}_{j}\}_{j=1}^{M}$を得る.その後,コスト関数$C_{\mathrm{pose}}(\cdot)$によって算出した姿勢類似性を利用して,トラックと検出結果を対応付ける.}
                \label{fig:method_overview}
            \end{figure*}


        \subsection{姿勢特徴量の抽出}
            ある時刻において,検出結果の集合$\mathcal{Y}=\{\mathbf{y}_j\}_{j=1}^{N}$に対して得られる姿勢特徴量の集合を$\mathcal{F} = \{\mathbf{f}_{j}\}_{j=1}^{N}$とすると,検出結果$\mathbf{y}_{j}$の姿勢特徴量$\mathbf{f}_{j}$は,次のように定義される.
                \begin{equation}
                    \mathbf{f}_{j} = \mathrm{GAP}(F_{j}) \in \mathbb{R}^{C} 
                \end{equation}
            ここで,$F_{j} \in \mathbb{R}^{C \times H \times W}$は,$\mathbf{y}_{j}$のBBoxに対応する画像領域をTop-Down方式の姿勢推定器に入力した際に,その中間層が出力する特徴マップである.この特徴マップ$F_{j}$は,対象物体$\mathbf{y}_{j}$の局所的な間接配置や身体構造を捉えており,外観が類似する物体が多く登場するシナリオでは,比較的頑健な物体表現が得られることが期待される.
            また,$\mathbf{f}_{j} \in \mathbb{R}^{C}$は,$F_{j}$にGAP(Global Average Pooling)を適用し,$C$次元のベクトルに変換することで得られる.

        \subsection{姿勢特徴を考慮したAssociation}
            ある時刻において,トラックの集合$\hat{\mathcal{X}} = \{\hat{\mathbf{x}}_{i}\}_{i=1}^{M}$が保持する姿勢情報の集合を$\mathcal{G} = \{\mathbf{g}_{i}, c_{i}\}_{i=1}^{M}$とする.ここで,$\mathbf{g}_{i} \in \mathbb{R}^{C}$はトラック$\hat{\mathbf{x}}_{i}$が持つ姿勢特徴量であり,$c_{i} \in [0,1]$はその姿勢信頼度を表す.集合$\mathcal{G}$は,Associationの結果に応じて時間的に更新される(後述).

            本研究では,姿勢特徴を考慮したAssociationを行うにあたり,トラックと検出結果の姿勢類似度を式\ref{eq:cost_pose}で定まる姿勢コスト関数$C_{\mathrm{pose}}(\cdot)$を用いて求める.
                \begin{equation}
                    C_{\mathrm{pose}}(\hat{\mathbf{x}}_{i}, \mathbf{y}_{j}) = \dfrac{\mathbf{g}_{i}^{\mathrm{T}} \mathbf{f}_{j}}{{\left\Vert \mathbf{g}_{i} \right\Vert}_{2}\,{\left\Vert \mathbf{f}_{j} \right\Vert}_{2}} \in [0, 1]
                    \label{eq:cost_pose}
                \end{equation}
            式\ref{eq:cost_pose}は$\mathbf{g}_{i}$と$\mathbf{f}_{j}$のはコサイン類似度を表し,その値が$1$に近づくほどトラック$\hat{\mathbf{x}_{i}}$と検出結果$\mathbf{y}_{j}$の姿勢類似性が高いことを意味する.
            最終的なコスト関数$\mathcal{L}_{\mathrm{match}}(\cdot)$は,OC-SORTの既存コスト関数$C_{\mathrm{ocsort}}(\cdot)$に姿勢コスト関数$C_{\mathrm{pose}}(\cdot)$を加えて式\ref{eq:final_cost}で与える.
                \begin{equation}
                    \mathcal{L}_{\mathrm{match}} (\hat{\mathbf{x}}_{i}, \mathbf{y}_{j}) = C_{\mathrm{ocsort}}(\hat{\mathbf{x}}_{i}, \mathbf{y}_{j}) + \lambda_{p} c_{i} C_{\mathrm{pose}}(\hat{\mathbf{x}}_{i}, \mathbf{y}_{j})
                    \label{eq:final_cost}
                \end{equation}
            ここで,$C_{\mathrm{ocsort}}(\cdot)$はIoUと速度方向の類似性に基づくコスト関数であり,$\lambda_{p}$は姿勢項の寄与を制御する重みである.また,$c_{i}$を掛けることで,トラック$\hat{\mathbf{x}}_{i}$が信頼度の低い姿勢特徴量$\mathbf{g}_{i}$を持つ場合には,$\mathbf{g}_{i}$から算出される姿勢項の影響を抑える狙いがある.実装上は,$\mathcal{L}_{\mathrm{match}}(\cdot) \leftarrow - \mathcal{L}_{\mathrm{match}}(\cdot)$として,ハンガリアン法によって式\ref{eq:Association}で表される総コストの最小化問題を解く.

        \subsection{姿勢情報の更新}
            ある時刻において,式\ref{eq:final_cost}のコスト関数$\mathcal{L}_{\mathrm{match}}(\cdot)$を用いたAssociationの結果,トラック$\hat{\mathbf{x}}_{i}$と検出結果$\mathbf{y}_{j}$が対応付けられたとする.このとき,$\hat{\mathbf{x}}_{i}$が保持する姿勢情報$\{ \mathbf{g}_{i}, c_{i} \}$は次のように更新する.
                \begin{equation}
                    \mathbf{g}_{i} \leftarrow \mathbf{f}_{j}, \quad c_{i} \leftarrow 1.0
                    \label{eq:update_pose_info}
                \end{equation}
            これは,最新の検出結果が持つ姿勢特徴量を対応付けられたトラックが引継ぎ,信頼度を最大値に戻す操作である.一方,どの検出結果にも対応付けられなかったトラック$\hat{\mathbf{x}}_{k}$については,保持している姿勢特徴量$\mathbf{g}_{k}$を維持しつつ,姿勢信頼度$c_{k}$のみを時間減衰させる.
                \begin{equation}
                    \mathbf{g}_{k} \leftarrow \mathbf{g}_{k}, \quad c_{k} \leftarrow \1_{ \{\gamma c_{k} > \tau\} } \gamma c_{k}
                    \label{eq:decay_pose_conf}
                \end{equation}
            ここで,$\gamma \in (0, 1]$は減衰率,$\tau$は姿勢信頼度の閾値であり,$\1_{ \{\gamma c_{k} > \tau\} }$は減衰値$\gamma c_{k}$が閾値$\tau$以上の場合に$1$,それ以外の場合に$0$を出力する指示関数である.式\ref{eq:decay_pose_conf}による信頼度の減衰挙動を図\ref{fig:pose_conf_decay}に示す.式\ref{eq:update_pose_info},\ref{eq:decay_pose_conf}に基づいて集合$\mathcal{G}$を更新することで,時間的に物体の姿勢が変化する中で,長期間対応付けられていないトラックの姿勢情報がAssociationに与える影響を抑制し,追跡精度の低下を防ぐ狙いがある.     
                \begin{figure}[tb]
                    \centering
                    \includegraphics[width=\linewidth]{figures/section3/conf_graph.pdf}
                    \caption{姿勢信頼度の減衰挙動.縦軸は姿勢信頼度,横軸はトラックが連続で検出結果と対応付けられなかった時間ステップ数を表す.}
                    \label{fig:pose_conf_decay}
                \end{figure}


    % 文献情報を条件分岐(サブファイルが直接コンパイルされた場合のみ適用)
    \ifSubfilesClassLoaded{
        \bibliography{bibliography/sample} % BibTeX ファイル名を指定(拡張子は不要)
        \bibliographystyle{ipsjunsrt} % 文献スタイルを指定
    }{}

\end{document}