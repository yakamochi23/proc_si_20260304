\documentclass[../main.tex]{subfiles} % 親ファイルを参照


\begin{document}
        
    \section{提案手法}

        本研究では,を提案する.具体手には,.さらに,することで必要としない.

        狙い.なぜいいのか?

        OC-SORTの上に設計することをの述べる.

        全体図を示す.

        \subsection{姿勢特徴量の抽出}
            ある時刻において,検出結果の集合$\hat{\mathcal{Y}}=\{\hat{\mathbf{y}}_j\}_{j=1}^{N}$に対して得られる姿勢特徴量の集合を$\mathcal{F} = \{\mathbf{f}_{j}\}_{j=1}^{N}$とすると,検出結果$\hat{\mathbf{y}}_{j}$の姿勢特徴量$\mathbf{f}_{j}$は,次のように定義される.
                \begin{equation}
                    \mathbf{f}_{j} = GAP(F_{j}) \in \mathbb{R}^{C} 
                \end{equation}
            ここで,$F_{j} \in \mathbb{R}^{C \times H \times W}$は$\hat{\mathbf{y}}_{j}$のBounding-Boxで切り取られるフレーム領域をTop-Down方式の姿勢推定モデルに入力した際に,その中間層が出力する特徴マップである.この特徴マップ$F_{j}$は,を捉えており,が期待される.
            また,$\mathbf{f}_{j} \in \mathbb{R}^{C}$は,$F_{j}$をGlobal Average Poolingによって変換した$C$次元のベクトルである.

        \subsection{姿勢特徴を考慮したAssociation}
            ある時刻において,トラックの集合$\mathcal{X} = \{\mathbf{x}_{j}\}_{i=1}^{M}$が保持する姿勢特徴量の集合を$\mathcal{G} = \{\mathbf{g}_{i}\}_{i=1}^{M}$とすると,Associationに導入する姿勢コスト$C_{\mathrm{Pose}}(\cdot)$は次式のように定式化される.
                \begin{equation}
                    C_{\mathrm{Pose}}(\mathbf{x}_{i}, \mathbf{y}_{j}) = \dfrac{\mathbf{g}_{i}^{\mathrm{T}} \mathbf{f}_{j}}{{\left\Vert \mathbf{g}_{i} \right\Vert}_{2}{\left\Vert \mathbf{f}_{j} \right\Vert}_{2}} \in [0, 1]
                \end{equation}
            $C_{\mathrm{Pose}}(\cdot)$はコサイン類似度を表し,その値が大きいほどトラック$\mathbf{x}_{i}$と検出結果$\mathbf{y}_{j}$の姿勢類似性が高いことを意味する.実装では,ハンガリアン法によってコストの総和を最小化する割当問題を解くため,$C_{\mathrm{Pose}} \leftarrow - C_{\mathrm{Pose}}$に変換して扱う.

        総コスト
            \begin{equation}
                \mathcal{L}_{\mathrm{match}} (\hat{\mathbf{x}}, \hat{\mathbf{y}}) = C_{\mathrm{IoU}}(\hat{\mathbf{x}}, \hat{\mathbf{y}}) + \lambda_{a} C_{\mathrm{Ang}}(\hat{\mathbf{x}}, \hat{\mathbf{y}}) + \lambda_{p} C_{\mathrm{Pose}}(\hat{\mathbf{x}}, \hat{\mathbf{y}})
            \end{equation}

        コスト$\mathcal{L}$を用いたAssociationにより,トラック$\mathbf{x}_{i}$と検出結果$\mathbf{y}_{j}$が対応付けられた場合,
            \begin{equation}
                \mathbf{g}_{i} \leftarrow \mathbf{f}_{j}, \quad c_{i} \leftarrow 1.0
            \end{equation}
        として,トラック$\mathbf{x}_{i}$が保持する姿勢特徴量$\mathbf{g}_{i}$および姿勢信頼度$c_{i}$の更新を行う.一方,どの検出結果にも対応付けられなかったトラック$\mathbf{x}_{k}$に関しては,姿勢特徴量$\mathbf{g}_{k}$の更新は行わずに,姿勢信頼度$c_{k}$を次のように減衰させる.
            \begin{equation}
                c_{k} \leftarrow \1_{ \{\gamma c_{k} > \tau\} } \gamma c_{k}
            \end{equation}
        ここで,$\gamma \in (0, 1]$は減衰率であり,ハイパーパラメータとして設定される.また,$\1_{ \{\gamma c_{k} > \tau\} }$は減衰値$\gamma c_{k}$が閾値$\tau$以上の場合に$1$,それ以外の場合に$0$を出力する指示関数である.これにより,


    % 文献情報を条件分岐(サブファイルが直接コンパイルされた場合のみ適用)
    \ifSubfilesClassLoaded{
        \bibliography{bibliography/sample} % BibTeX ファイル名を指定(拡張子は不要)
        \bibliographystyle{ipsjunsrt} % 文献スタイルを指定
    }{}

\end{document}