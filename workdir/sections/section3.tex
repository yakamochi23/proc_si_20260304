\documentclass[../main.tex]{subfiles} % 親ファイルを参照


\begin{document}
        
    \section{提案手法}

        本研究では,を提案する.具体手には,.さらに,することで必要としない.

        ある時刻において,検出結果の集合$\hat{\mathcal{Y}}=\{\hat{\mathbf{y}}_j\}_{j=1}^{N}$に対して得られる姿勢特徴量の集合$\mathcal{F}$を次のように定義する.
            \begin{equation}
                \mathcal{F} = \{ \mathbf{f}_{j} \}_{j=1}^{N} = \{GAP(F_{j})\}_{j=1}^{N}
            \end{equation}
        ここで,$F_{j} \in \mathbb{R}^{C \times H \times W}$は$\hat{\mathbf{y}}_{j}$のBounding-Box(幅を$W$,高さを$H$とする)で切り取られるフレーム領域をTop-Down方式の姿勢推定モデルに入力際に得られる中間特徴量である.中間特徴量$F_{j}$は,期待される.
        また,$\mathbf{f}_{j} = GAP(F_{j}) \in \mathbb{R}^{C}$は,$F_{j}$をGlobal Average Poolingによって変換した$C$次元のベクトルである.


        狙い.なぜいいのか?
            \begin{equation}
                \mathcal{L}_{\mathrm{match}} (\hat{\mathbf{x}}, \hat{\mathbf{y}}) = C_{\mathrm{IoU}}(\hat{\mathbf{x}}, \hat{\mathbf{y}}) + \lambda_{a} C_{\mathrm{Ang}}(\hat{\mathbf{x}}, \hat{\mathbf{y}}) + \lambda_{p} C_{\mathrm{Pose}}(\hat{\mathbf{x}}, \hat{\mathbf{y}})
            \end{equation}
        ここで,
            \begin{equation}
                C_{\mathrm{Pose}} (\hat{\mathbf{x}}, \hat{\mathbf{y}}) = - \dfrac{\mathbf{f}_{i} \cdot \mathbf{f}_{j}}{\mathbf{f}_{i} \mathbf{f}_{j} }
                \label{eq:C_pose}
            \end{equation}
        ここで,$f_{i} \in \mathbb{R}^{d}$

        更新方法


    % 文献情報を条件分岐(サブファイルが直接コンパイルされた場合のみ適用)
    \ifSubfilesClassLoaded{
        \bibliography{bibliography/sample} % BibTeX ファイル名を指定(拡張子は不要)
        \bibliographystyle{ipsjunsrt} % 文献スタイルを指定
    }{}

\end{document}