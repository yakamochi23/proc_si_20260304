\documentclass[../main.tex]{subfiles} % 親ファイルを参照


\begin{document}
        
    \section{提案手法}

        本研究では,を提案する.具体手には,.さらに,することで必要としない.

        狙い.なぜいいのか?

        OC-SORTの上に設計することをの述べる.

        全体図を示す.
        \subsection{OC-SORT}
            あいうえお.

        \subsection{姿勢特徴量の抽出}
            ある時刻において,検出結果の集合$\hat{\mathcal{Y}}=\{\hat{\mathbf{y}}_j\}_{j=1}^{N}$に対して得られる姿勢特徴量の集合を$\mathcal{F} = \{\mathbf{f}_{j}\}_{j=1}^{N}$とすると,検出結果$\hat{\mathbf{y}}_{j}$の姿勢特徴量$\mathbf{f}_{j}$は,次のように定義される.
                \begin{equation}
                    \mathbf{f}_{j} = \mathrm{GAP}(F_{j}) \in \mathbb{R}^{C} 
                \end{equation}
            ここで,$F_{j} \in \mathbb{R}^{C \times H \times W}$は,$\hat{\mathbf{y}}_{j}$のBounding-Boxで切り取られるフレーム領域をTop-Down方式の姿勢推定器に入力した際に,その中間層が出力する特徴マップである.この特徴マップ$F_{j}$は,対象物体$\hat{\mathbf{y}}_{j}$の局所的な間接配置や身体構造を捉えており,外観が類似する物体が多く登場するシナリオでは,比較的頑健な物体表現が得られることが期待される.
            また,$\mathbf{f}_{j} \in \mathbb{R}^{C}$は,$F_{j}$にGAP(Global Average Pooling)を適用し,$C$次元のベクトルに変換することで得られる.

        \subsection{姿勢特徴を考慮したAssociation}
            ある時刻において,トラックの集合$\mathcal{X} = \{\mathbf{x}_{i}\}_{i=1}^{M}$が保持する姿勢情報の集合を$\mathcal{G} = \{\mathbf{g}_{i}, c_{i}\}_{i=1}^{M}$とする.ここで,$\mathbf{g}_{i} \in \mathbb{R}^{C}$はトラック$\mathbf{x}_{i}$が持つ姿勢特徴量であり,$c_{i} \in [0,1]$はその姿勢信頼度を表す.集合$\mathcal{G}$は,Associationの結果に応じて時間的に更新される(後述).

            本研究では,姿勢特徴を考慮したAssociationを行うにあたり,トラックと検出結果の姿勢類似度を式\ref{eq:cost_pose}で定まる姿勢コスト関数$C_{\mathrm{pose}}(\cdot)$を用いて求める.
                \begin{equation}
                    C_{\mathrm{pose}}(\mathbf{x}_{i}, \hat{\mathbf{y}}_{j}) = \dfrac{\mathbf{g}_{i}^{\mathrm{T}} \mathbf{f}_{j}}{{\left\Vert \mathbf{g}_{i} \right\Vert}_{2}\,{\left\Vert \mathbf{f}_{j} \right\Vert}_{2}} \in [0, 1]
                    \label{eq:cost_pose}
                \end{equation}
            式\ref{eq:cost_pose}は$\mathbf{g}_{i}$と$\mathbf{f}_{j}$のはコサイン類似度を表し,その値が$1$に近づくほどトラック$\mathbf{x}_{i}$と検出結果$\hat{\mathbf{y}}_{j}$の姿勢類似性が高いことを意味する.
            最終的なコスト関数$\mathcal{L}_{\mathrm{match}}(\cdot)$は,OC-SORTの既存コスト関数$C_{\mathrm{ocsort}}(\cdot)$に姿勢コスト関数$C_{\mathrm{pose}}(\cdot)$を加えて式\ref{eq:final_cost}で与える.
                \begin{equation}
                    \mathcal{L}_{\mathrm{match}} (\hat{\mathbf{x}}_{i}, \hat{\mathbf{y}}_{j}) = C_{\mathrm{ocsort}}(\hat{\mathbf{x}}_{i}, \hat{\mathbf{y}}_{j}) + \lambda_{p} c_{i} C_{\mathrm{pose}}(\hat{\mathbf{x}}_{i}, \hat{\mathbf{y}}_{j})
                    \label{eq:final_cost}
                \end{equation}
            ここで,$C_{\mathrm{ocsort}}(\cdot)$はIoUと速度方向の類似性に基づくコスト関数であり,$\lambda_{p}$は姿勢項の寄与を制御する重みである.また,$c_{i}$を掛けることで,トラック$\mathbf{x}_{i}$が信頼度の低い姿勢特徴量$\mathbf{g}_{i}$を持つ場合には,$\mathbf{g}_{i}$から算出される姿勢項の影響を抑える狙いがある.実装上は,$\mathcal{L}_{\mathrm{match}}(\cdot) \leftarrow - \mathcal{L}_{\mathrm{match}}(\cdot)$として,ハンガリアン法によって式\ref{eq:Association}で表される総コストの最小化問題を解く.

        \subsection{姿勢情報の更新}
            ある時刻において,式\ref{eq:final_cost}のコスト関数$\mathcal{L}_{\mathrm{match}}(\cdot)$を用いたAssociationの結果,トラック$\mathbf{x}_{i}$と検出結果$\mathbf{y}_{j}$が対応付けられたとする.このとき,$\mathbf{x}_{i}$が保持する姿勢情報$\{ \mathbf{g}_{i}, c_{i} \}$は次のように更新する.
                \begin{equation}
                    \mathbf{g}_{i} \leftarrow \mathbf{f}_{j}, \quad c_{i} \leftarrow 1.0
                    \label{eq:update_pose_info}
                \end{equation}
            これは,最新の検出結果が持つ姿勢特徴量を対応付けられたトラックが引継ぎ,信頼度を最大値に戻す操作である.一方,どの検出結果にも対応付けられなかったトラック$\mathbf{x}_{k}$については,保持している姿勢特徴量$\mathbf{g}_{k}$を維持しつつ,姿勢信頼度$c_{k}$のみを時間減衰させる.
                \begin{equation}
                    \mathbf{g}_{k} \leftarrow \mathbf{g}_{k}, \quad c_{k} \leftarrow \1_{ \{\gamma c_{k} > \tau\} } \gamma c_{k}
                    \label{eq:decay_pose_conf}
                \end{equation}
            ここで,$\gamma \in (0, 1]$は減衰率,$\tau$は姿勢信頼度の閾値であり,$\1_{ \{\gamma c_{k} > \tau\} }$は減衰値$\gamma c_{k}$が閾値$\tau$以上の場合に$1$,それ以外の場合に$0$を出力する指示関数である.式\ref{eq:decay_pose_conf}による信頼度の減衰挙動を図\ref{fig:pose_conf_decay}に示す.式\ref{eq:update_pose_info},\ref{eq:decay_pose_conf}に基づいて集合$\mathcal{G}$を更新することで,時間的に物体の姿勢が変化する中で,長期間対応付けられていないトラックの姿勢情報がAssociationに与える影響を抑制し,追跡精度の低下を防ぐ狙いがある.      


    % 文献情報を条件分岐(サブファイルが直接コンパイルされた場合のみ適用)
    \ifSubfilesClassLoaded{
        \bibliography{bibliography/sample} % BibTeX ファイル名を指定(拡張子は不要)
        \bibliographystyle{ipsjunsrt} % 文献スタイルを指定
    }{}

\end{document}