\usepackage[top=20mm,bottom=20mm,left=20mm,right=20mm]{geometry}
%% jsclasses系で文字サイズ11pt や 12pt をクラスオプションに指定すると、
%% 長さが拡大されるため、nomagオプションを併用している.
%% https://oku.edu.mie-u.ac.jp/~okumura/jsclasses/ のFAQをよく読むこと.


%\usepackage{layout}
\usepackage[T1]{fontenc} %欧文エンコーディング指定
\usepackage{lmodern}     %欧文フォント指定
%\usepackage{styles/opucb4thesis}%%%  卒業論文の場合.
% opucb4thesis.sty が同フォルダに必要

% CloudLaTeXの場合は下の1行を有効にすること
% \AtBeginDvi{\special{pdf:mapfile ptex-ipaex.map}}

\usepackage{amsmath,amssymb,bm}% 数式用
\usepackage{graphicx} %dvipdfmx を前提としている
\usepackage{float} % 図表の位置決め用
\usepackage{url}      % URL等記載用.\verbより便利
% ソースコード、等幅文書貼り付け用
\usepackage{framed}
\setlength{\OuterFrameSep}{0pt}
\usepackage{moreverb}
% 必要に応じて plistings.sty を入手して使うとよい。
% https://github.com/h-kitagawa/plistings

\usepackage{xcolor} % 色を使うために必要
\usepackage{listings,jlisting}%ソースコードを載せるための環境
\renewcommand{\lstlistingname}{リスト}
\lstset{%listingsの設定
    language = {Python},%プログラム言語 C, Python
    backgroundcolor={\color[gray]{.9}},%背景色と透過度
    breaklines = true,%枠外に行った時の自動改行有無
    breakindent = 10pt,%自動改行後のインデント量(デフォルトでは20[pt])
    columns=[l]{fullflexible},%文字間の余白
    basicstyle = {\small},%標準のフォント
    identifierstyle={\small},% 変数名などのフォントを設定
    commentstyle = {\small\itshape\color[cmyk]{1,0.4,1,0}},%コメントのフォント
    classoffset = 0,%関数名等の色の設定
    keywordstyle = {\small\bfseries\color[rgb]{0,0.3,0.9}},%キーワード(int, ifなど)のフォント
    ndkeywordstyle={\small\color[cmyk]{0,1,0,0}}, % keywordstyle以外に定義されている予約語のフォントを設定
    stringstyle = {\ttfamily\color[rgb]{0.83,0.47,0.2}},%表示する文字列のフォント
    showstringspaces=false,%文字列の空白記号の有無(デフォルトはtrue)
    %枠 "t"は上に線を記載, "T"は上に二重線を記載
	%他オプション:leftline,topline,bottomline,lines,single,shadowbox
    frame = tRBl,
    framesep = 5pt,%frameまでの間隔(行番号とプログラムの間)
    numbers = left,%行番号の位置
    stepnumber = 1,%行番号の間隔
    numberstyle = {\scriptsize},%行番号のフォント
    tabsize = 4,%タブの大きさ
    xrightmargin=0zw,%右側の余白
    xleftmargin=3zw,%左側の余白
    captionpos = t%キャプションの場所("tb"ならば上下両方に記載)
}



\usepackage{subfiles}
\renewcommand{\baselinestretch}{1.0} % 全体の行間を変更(デフォルトは倍率1.0, 先生に提出する際は2.0くらいが良い)
\usepackage[subrefformat=parens]{subcaption}% 複数の図を貼るときのsubcaption
\renewcommand\thesubfigure{(\alph{subfigure})}% 図版のフォーマット用
\captionsetup[subfigure]{labelformat=simple, labelsep=space}% 図版のフォーマット用